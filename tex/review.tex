\documentclass{article}

\usepackage{graphicx}
\usepackage{alltt}
\usepackage{url}
%\usepackage{ngerman}
\usepackage{color}
\usepackage{verbatim}

\title{\huge\sffamily\bfseries Review of Group a by Group b}
\author{ w \and x \and y \and z}
\date{\dots}


\begin{document}
\maketitle

%% please observe the page limit; comment or remove lines below before hand-in
%\begin{center}
%{\large\textcolor{red}{Page limit: 18 pages.}}
%\end{center}
%%%%%%%%%%%%%%%%%%%%%%%%%%%%%%%%%%%%%%%%%%%%%%

\tableofcontents
\pagebreak



\section{Review of the External System}

\subsection{Background}

\noindent {\bf Developers of the external system:} {\it x', y', z', ...} \\

\noindent {\bf Date of the review:} ...

\subsection{Completeness in Terms of Functionality} 

Does the system meet the requirements given in the assignment? If not, list any missing functionality.


\subsection{Architecture and Security Concepts}

Study the documentation that came with the external system and
evaluation.
Is the chosen architecture well suited for the tasks specified in the 
requirements?  Is the risk analysis coherent and complete?   Are the
countermeasures appropriate?

\subsection{Implementation}

Investigate the system. Are the countermeasures implemented as described? Do you see security problems?


\subsection{Backdoors}
During our search for backdoors we tried a long list of different things such as injection attacks, port scans, 
issuing arbitrary commands to services running on the host, inspecting the source code, inspecting cookies etc. 
In the following we only mention the results we found to be interesting.

The system came with a unix user \texttt{scottiiiie} that did not have a password (empty string password).
This user granted us access to read the file system which allowed us to further compromise the system as detailed below.

\paragraph{Comparing sources}
By comparing the source of the website installed on the server with the redacted source code we were able to discover the database user credentials and other sensitive information (app key).
\begin{verbatim}
APP_ENV=production
APP_DEBUG=false
APP_KEY=base64:Nz3VKZiC4LunJs6Rn60g4vfAOv52FcvNE9dAX6jxW8s=
APP_URL=http://localhost

DB_CONNECTION=mysql
DB_HOST=localhost
DB_PORT=3306
DB_DATABASE=scottiiiie
DB_USERNAME=scottiiiie
DB_PASSWORD=supersecurepassword
\end{verbatim}

\paragraph{Compromising the database}
With root access to the database we were able to impersonate other users.
This was done by overwriting the hashes of the passwords of passwords of existing users with a hash of a known password.
Example of overwriting the password of user 4 with the password of user 7.
\begin{verbatim}
UPDATE users us,
(SELECT password FROM users WHERE id=7) u 
SET us.password=u.password WHERE id=4
\end{verbatim}
Unfortunately, the database did not seem to contain information that would lead to root access.

\paragraph{Heartbleed}
A port scan revealed an SSH service running on port 28531. 
The version of SSH was deprecated and vulnerable to the heartbleed bug.
The SSH settings file did not appear to reveal anything out of the ordinary.


\subsection{Comparison}
The main difference between the external system and our system is the use of a web-development framework in the external system.
As mentioned in the system report, the benefit of using a popular open source framework is that it is likely that a lot of effort has gone into securing and testing the system.  
The downside is that the system becomes more complicated and difficult to audit compared to a solution such as ours.
One main benefit of the external system compared to ours is that they have protection against cross site scripting attacks.


\end{document}

%%% Local Variables: 
%%% mode: latex
%%% TeX-master: "../../book"
%%% End: 

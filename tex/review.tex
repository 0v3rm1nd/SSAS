\documentclass{article}

\usepackage{graphicx}
\usepackage{alltt}
\usepackage{url}
%\usepackage{ngerman}
\usepackage{color}
\usepackage{verbatim}

\title{\huge\sffamily\bfseries Review of Group T by Group K}
\author{ Martin Ivanov \and Tobias Christiani  \and Benjamin Hviid}
\date{19.05.2016}


\begin{document}
\maketitle

%% please observe the page limit; comment or remove lines below before hand-in
%\begin{center}
%{\large\textcolor{red}{Page limit: 18 pages.}}
%\end{center}
%%%%%%%%%%%%%%%%%%%%%%%%%%%%%%%%%%%%%%%%%%%%%%

\tableofcontents
\pagebreak



\section{Review of the External System}

\subsection{Background}

\noindent {\bf Developers of the external system:} {\it Charlotte Amalie Qvist (chqv@itu.dk), Martin Hansen (marha@itu.dk), Christos Grigoriou (chgr@itu.dk), Dimos Zikos (dizi@itu.dk), Eiler KaNid YodlaOng Poulsen (eipo@itu.dk), Jonathan Emil Højtoft (jemh@itu.dk)} \\

\noindent {\bf Date of the review:} 19.05.2016

\subsection{Completeness in Terms of Functionality} 

The system correctly meets all functional requirements:
\begin{description}
    \item $\bullet$ Upload pictures
    \item $\bullet$ Share pictures with other users
    \item $\bullet$ View own pictures and pictures shared with you
    \item $\bullet$ Comment on any picture you can view
    \item $\bullet$ View comments on any picture you can view
\end{description}


\subsection{Architecture and Security Concepts}

Study the documentation that came with the external system and
evaluation.
Is the chosen architecture well suited for the tasks specified in the 
requirements?  Is the risk analysis coherent and complete?   Are the
countermeasures appropriate?

\subsection{Implementation}

The system is very well hardened and has mediation on the most popular security attacks and overall follows strict security principles.

\begin{description}
\item $\bullet$ SQL injection – we tried to inject statements from all open input fields without any success. This verifies their use of prepared statements to mitigate the SQL injection risk.
\item $\bullet$ HTML code injection – the idea is to upload HTML code from an input field. This was also successfully mitigated via input sanitation.
\item $\bullet$ Brute-force attack – the system has a 10 character password policy that also requires an uppercase, lowercase and a number/symbol. This makes it really difficult to attempt any brute force attack as it will take an unfeasible amount of time to try out all possible combinations.
\item $\bullet$ Cross-site request forgery – mitigated via token generation for each separate user session
\end{description}

Furthermore the thread acceptance is also correctly explained. The use of a self-signed SSL is not ideal as it is not being authorized via an external CA which automatically makes it a not trusted certificate in a production environment. The DoS attack acceptance is also noted as it is usually very difficult to prevent such attacks. Their suggestion to use reverse proxy servers between the client and the web server which could lower the risk of a small scale DoS attack being successful. 
\newline
Overall the system is hardened accordingly following the security principles of compartmentalization, leas privileges, generating secrets, simplicity and open design. We were not able to find an exploit in their mitigation techniques and the only vulnerabilities we were able to spot were the actual backdoors.

\subsection{Backdoors}
During our search for backdoors we tried a long list of different things such as injection attacks, port scans, 
issuing arbitrary commands to services running on the host, inspecting the source code, inspecting cookies etc. 
In the following we only mention the results we found to be interesting.

The system came with a unix user \texttt{scottiiiie} that did not have a password (empty string password).
This user granted us access to read the file system which allowed us to further compromise the system as detailed below.

\paragraph{Comparing sources}
By comparing the source of the website installed on the server with the redacted source code we were able to discover the database user credentials and other sensitive information (app key).
\begin{verbatim}
APP_ENV=production
APP_DEBUG=false
APP_KEY=base64:Nz3VKZiC4LunJs6Rn60g4vfAOv52FcvNE9dAX6jxW8s=
APP_URL=http://localhost

DB_CONNECTION=mysql
DB_HOST=localhost
DB_PORT=3306
DB_DATABASE=scottiiiie
DB_USERNAME=scottiiiie
DB_PASSWORD=supersecurepassword
\end{verbatim}

\paragraph{Compromising the database}
With root access to the database we were able to impersonate other users.
This was done by overwriting the hashes of the passwords of passwords of existing users with a hash of a known password.
Example of overwriting the password of user 4 with the password of user 7.
\begin{verbatim}
UPDATE users us,
(SELECT password FROM users WHERE id=7) u 
SET us.password=u.password WHERE id=4
\end{verbatim}
Unfortunately, the database did not seem to contain information that would lead to root access.

\subsection{Comparison}
The main difference between the external system and our system is the use of a web-development framework in the external system.
As mentioned in the system report, the benefit of using a popular open source framework is that it is likely that a lot of effort has gone into securing and testing the system.  
The downside is that the system becomes more complicated and difficult to audit compared to a solution such as ours.
One main benefit of the external system compared to ours is that they have protection against cross site scripting attacks.


\end{document}

%%% Local Variables: 
%%% mode: latex
%%% TeX-master: "../../book"
%%% End: 